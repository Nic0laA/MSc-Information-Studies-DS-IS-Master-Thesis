\section{Discussion}
\label{sec:discussion}
% Compare your results with the state-of-the-art and reflect upon the results and limitations of the study. You can already hint at future work to which you come back in the conclusion section.
Write your discussion here. Do not forget to use sub-sections. Normally, the discussion starts with comparing your results to other studies as precisely as possible. The limitations should be reflected upon in terms such as reproducibility,  scalability,  generalizability,  reliability  and  validity. It is also important to mention ethical concerns.

The overall objective of this work is to increase the efficiency of the \texttt{Peregrine} data analysis pipeline. The work will begin with reproducing the results from papers~\cite{bhardwaj2023peregrine} and~\cite{alvey2023things}, as this will form the benchmark to which the eventual results will be compared to.

Many moving parts to problem - number of truncation rounds, the number of simulations per round, network architecture, the truncation, the sampling strategy of the priors.

We will investigate whether some more active learning can be introduced to increase the efficiency of the sampling process. For instance, some parameters such as the chirp mass\footnote{The chirp mass is a combination of the two object masses in the binary system, and is a key factor in the gravitational wave frequency as the two objects spiral inwards toward each other.\\$\mathcal{M} = \frac{(m_1 \cdot m_2)^{3/5}}{(m_1 + m_2)^{1/5}}$} may be inferred early on with relatively high confidence, but currently in successive simulation rounds continues to be sampled from a uniform distribution within $\pm5\sigma$~\cite{Miller_TMNRE_2021}. We will investigate whether it's possible to be more selective in the sampling of parameters that are known with relatively high confidence in an un-biased way. Therefore, we can focus the simulation budget more on the parameters we know with less confidence. To do this in a systematic way, a complete survey of how influential each parameter is on the different segments of the GW signal will be carried-out first.

