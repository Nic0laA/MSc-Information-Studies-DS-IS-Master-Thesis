\section{Results}
\label{sec:results}
% Give the outcomes for each research question in the form of a table or graphic (with caption).
% Write about your results here. Good captions to tables and/or figures are key.

\subsection{Transformer models}

% Show loss curve of UNet, ViT and MVTS transformer. 
In this section, we show the results during the pretraining of the transformer models

We can conclude that the ViT does not extract any additional features from the data, since the loss plateus to the same value. However, due to extra features the ViT is much slower. The mvts model is not a good model.

The pretrained ViT model was used for the evaluation of Peregrine. Round 2 training of ViT took 11 hours, so was terminated not to waste computing budget.

AUROC for 15 features after 24 hours of training.

\begin{table}
    \caption{Loss values for each of the features}
    \begin{tabular}{lrrr}
    \toprule
    feature & mvts & vit & unet \\
    \midrule
    mass ratio & -0.231 & -0.295 & -0.300 \\
    chirp mass & -0.542 & -0.676 & -0.701 \\
    theta jn & -0.327 & -0.463 & -0.426 \\
    phase & 0.000 & 0.000 & 0.000 \\
    tilt1 & -0.107 & -0.167 & -0.174 \\
    tilt2 & -0.018 & -0.036 & -0.036 \\
    a1 & -0.078 & -0.104 & -0.132 \\
    a2 & -0.005 & -0.008 & -0.007 \\
    phi12 & 0.000 & 0.000 & 0.000 \\
    phijl & -0.199 & -0.291 & -0.287 \\
    luminosity distance & -0.196 & -0.294 & -0.291 \\
    dec & -0.830 & -0.923 & -0.899 \\
    ra & -1.020 & -1.085 & -1.066 \\
    psi & -0.063 & -0.117 & -0.090 \\
    geocent time & -0.965 & -1.104 & -1.105 \\
    \bottomrule
    \end{tabular}
\end{table}

\begin{figure}
  \centering
  \includegraphics[width=1\linewidth]{media/images/Pretraining_loss_curve.png}
  \caption{Bla}
  \label{fig:pretain_loss_curve}
\end{figure}

\subsection{Peregrine Network}

Runtime for the rounds.

\begin{table}
\caption{Network performance of different network architectures for Peregrine}
\begin{tabular}{lrrrrrrr}
\toprule
\multicolumn{1}{p{0.5cm}}{\raggedright Network}  & 
\multicolumn{1}{p{0.5cm}}{\raggedleft Round}  &
\multicolumn{1}{p{1cm}}{\raggedleft Num \\ Epochs} &
\multicolumn{1}{p{1cm}}{\raggedleft Num \\ Steps} &
\multicolumn{1}{p{1cm}}{\raggedleft Sampling \\ Fraction} &
\multicolumn{1}{p{0.75cm}}{\raggedleft Train \\ Loss} &
\multicolumn{1}{p{0.75cm}}{\raggedleft Test \\ Loss} &
\multicolumn{1}{p{1cm}}{\raggedleft Avg \\ AUROC} \\
\midrule
UNet & 1 & 104 & 10504 & 4.98e-03 & -4.71 & -4.22 & 0.706 \\
UNet & 2 & 79 & 16432 & 2.66e-05 & -4.78 & -4.65 & 0.738 \\
UNet & 3 & 77 & 23793 & 2.38e-06 & -4.71 & -4.47 & 0.741 \\
UNet & 4 & 74 & 30858 & 1.61e-06 & -4.14 & -4.43 & 0.741 \\
UNet & 5 & 90 & 37530 & 8.15e-07 & -4.43 & -4.39 & 0.742 \\
UNet & 6 & 60 & 31440 & 6.03e-07 & -4.19 & -4.25 & 0.736 \\
\midrule
Att UNet & 1 & 74 & 7474 & 1.04e-02 & -4.48 & -4.16 & 0.710 \\
Att UNet & 2 & 78 & 16224 & 2.21e-05 & -4.83 & -4.65 & 0.731 \\
Att UNet & 3 & 58 & 17922 & 5.59e-06 & -4.35 & -4.35 & 0.735 \\
Att UNet & 4 & 54 & 22518 & 2.76e-06 & -4.32 & -4.27 & 0.736 \\
Att UNet & 5 & 87 & 36279 & 2.51e-06 & -4.38 & -4.16 & 0.735 \\
Att UNet & 6 & 83 & 43492 & 1.74e-06 & -4.09 & -4.19 & 0.734 \\
\midrule
UNet 5\% & 1 & 68 & 6868 & 4.21e-03 & -5.03 & -4.43 & 0.716 \\
UNet 5\% & 2 & 65 & 13520 & 5.40e-05 & -4.85 & -4.49 & 0.731 \\
UNet 5\% & 3 & 75 & 23175 & 7.71e-06 & -4.05 & -4.33 & 0.736 \\
UNet 5\% & 4 & 48 & 20016 & 4.67e-06 & -4.09 & -4.41 & 0.743 \\
UNet 5\% & 5 & 60 & 25020 & 3.91e-06 & -4.27 & -4.46 & 0.743 \\
UNet 5\% & 6 & 30 & 15720 & 3.52e-06 & -4.74 & -4.38 & 0.741 \\
\midrule
UNet 10\% & 1 & 39 & 3939 & 4.31e-02 & -3.98 & -3.89 & 0.689 \\
UNet 10\% & 2 & 35 & 7280 & 3.72e-03 & -3.67 & -3.76 & 0.706 \\
UNet 10\% & 3 & 58 & 17922 & 4.33e-05 & -4.42 & -4.62 & 0.732 \\
UNet 10\% & 4 & 57 & 23769 & 1.14e-05 & -4.54 & -4.27 & 0.735 \\
UNet 10\% & 5 & 30 & 12510 & 8.17e-06 & -3.90 & -3.96 & 0.725 \\
UNet 10\% & 6 & 56 & 29344 & 4.87e-06 & -4.08 & -4.31 & 0.738 \\
\midrule
ViT & 1 & 30 & 12540 & 4.37e-05 & -6.15 & -5.58 & 0.753 \\
ViT & 2 & 60 & 50460 & 2.03e-06 & -5.82 & -5.28 & 0.757 \\
\bottomrule
\end{tabular}
\end{table}


Look at different parameters for round 6.
Compare the four different UNet models.

\begin{table}
\caption{Round 6 AUC and loss values}
\hspace*{-2cm}
\begin{tabular}{lrrrrrrrr}
\toprule
Param & 
\multicolumn{1}{p{0.3cm}}{\raggedright UNet}  & 
\multicolumn{1}{p{0.3cm}}{\raggedleft Att UNet}  &
\multicolumn{1}{p{0.3cm}}{\raggedleft UNet 10\%}  &
\multicolumn{1}{p{0.3cm}}{\raggedleft UNet 5\%}  &
\multicolumn{1}{p{0.3cm}}{\raggedright UNet}  & 
\multicolumn{1}{p{0.3cm}}{\raggedleft Att UNet}  &
\multicolumn{1}{p{0.3cm}}{\raggedleft UNet 10\%}  &
\multicolumn{1}{p{0.3cm}}{\raggedleft UNet 5\%}  \\
\midrule
\( q \) & 0.765 & 0.756 & 0.744 & 0.761 & -0.282 & -0.257 & -0.246 & -0.275 \\
\( M \) & 0.833 & 0.817 & 0.825 & 0.817 & -0.448 & -0.403 & -0.428 & -0.415 \\
\( \theta_{jn} \) & 0.800 & 0.795 & 0.815 & 0.812 & -0.352 & -0.343 & -0.391 & -0.386 \\
\( \phi_c \) & 0.499 & 0.500 & 0.501 & 0.506 & 0.000 & 0.000 & 0.000 & -0.000 \\
\( \theta_1 \) & 0.756 & 0.754 & 0.746 & 0.748 & -0.243 & -0.240 & -0.221 & -0.226 \\
\( \theta_2 \) & 0.638 & 0.640 & 0.627 & 0.628 & -0.069 & -0.069 & -0.055 & -0.058 \\
\( a_1 \) & 0.725 & 0.714 & 0.715 & 0.722 & -0.193 & -0.180 & -0.170 & -0.187 \\
\( a_2 \) & 0.573 & 0.571 & 0.567 & 0.579 & -0.022 & -0.023 & -0.019 & -0.023 \\
\( \phi_{12} \) & 0.499 & 0.498 & 0.502 & 0.503 & 0.000 & 0.001 & 0.000 & -0.000 \\
\( \phi_{jl} \) & 0.856 & 0.892 & 0.890 & 0.901 & -0.503 & -0.620 & -0.594 & -0.633 \\
\( d_L \) & 0.828 & 0.812 & 0.814 & 0.818 & -0.444 & -0.400 & -0.394 & -0.411 \\
\( \delta \) & 0.848 & 0.846 & 0.856 & 0.859 & -0.496 & -0.480 & -0.506 & -0.519 \\
\( \alpha \) & 0.858 & 0.841 & 0.854 & 0.853 & -0.520 & -0.468 & -0.509 & -0.501 \\
\( \psi \) & 0.714 & 0.715 & 0.741 & 0.746 & -0.175 & -0.177 & -0.217 & -0.231 \\
\( t_c \) & 0.854 & 0.865 & 0.873 & 0.854 & -0.515 & -0.545 & -0.565 & -0.511 \\

\bottomrule
\end{tabular}
\end{table}

\subsection{Peregrine Run Strategy}

% Network reinitialisation
% Sampling strategy
% Simulation scheduling

% Compare posteriors of 

% Sometimes,  especially  if  you  have  quite  different experiments or research  questions,  it makes sense to interleave the experimental setup and the results sections, so the reader does not get lost. It is then helpful to structure clearly in (sub)subsections.