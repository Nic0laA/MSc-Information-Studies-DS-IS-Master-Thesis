\section{Risk Assessment}
\label{sec:risk_assessment}
% Describe the risks which you could run into and how you will mitigate them.

The thesis utilises simulated data (not real gravitational wave data) since we are focused on the optimisation of the data analysis pipeline itself, and not the physics of astronomical objects. The `ground-truth' data used for validation of the thesis findings will be the results of papers \cite{bhardwaj2023peregrine} and \cite{alvey2023things}. The risks of using simulated data are mitigated since the data is generated using a method that is well established and accepted by the gravitational wave community for this purpose \cite{alvey2023things}. In addition, if we were to use real experimental data, then we can not know for certain what the true values of the inferred parameters are, making the overall validation less reliable.

Another potential risk is that the required computational effort could exceed the assigned student budget provided for the Snellius cluster. This risk can be mitigated by monitoring the consumed budget throughout the project, or possibly requesting more budget.
